\documentclass{article}
\usepackage[UTF8]{ctex}
\usepackage{pythonhighlight}
\usepackage{markdown}
\usepackage{listings}
% \lstset{
%     basicstyle=\tt,
%     keywordstyle=\color{purple}\bfseries,
%     identifierstyle=\color{brown!80!black},
%     commentstyle=\color{gray}
%     showstringspaces=false,
%     backgroundcolor=\color[RGB]{245,245,244},
% }
\lstset{
    basicstyle          =   \tt,          % 基本代码风格
    identifierstyle=\color{brown!80!black},
    keywordstyle        =   \color{purple}\bfseries,          % 关键字风格
    commentstyle        =   \rmfamily\itshape,  % 注释的风格,斜体
    stringstyle         =   \ttfamily,  % 字符串风格
    flexiblecolumns,                % 别问为什么,加上这个
    numbers             =   left,   % 行号的位置在左边
    showspaces          =   false,  % 是否显示空格,显示了有点乱,所以不现实了
    numberstyle         =   \zihao{-5}\ttfamily,    % 行号的样式,小五号,tt等宽字体
    showstringspaces    =   false,
    captionpos          =   t,      % 这段代码的名字所呈现的位置,t指的是top上面
    frame               =   lrtb,   % 显示边框
    backgroundcolor=\color[RGB]{245,245,244},
}

% Language setting
% Replace `english' with e.g. `spanish' to change the document language
\usepackage[english]{babel}
\usepackage{float}
% Set page size and margins
% Replace `letterpaper' with `a4paper' for UK/EU standard size
\usepackage[letterpaper,top=2cm,bottom=2cm,left=3cm,right=3cm,marginparwidth=1.75cm]{geometry}

% Useful packages
\usepackage{amsmath}
\usepackage{graphicx}
\usepackage[colorlinks=true, allcolors=blue]{hyperref}

\title{数逻实验报告Lab6}
\author{雷远航}

\begin{document}

\maketitle

\begin{abstract}
    实验项目:七段数码管
\end{abstract}

\section{操作方法与实验步骤}

\subsection{原理图设计实现显示译码模块MyMC14495}
\subsubsection{绘制原理图}
\subsubsection*{根据真值表绘制出七位译码器}
    

\subsubsection{对生成的原理图进行检验}
完成Synthesize-XST,Implement Design,Manage Configuration Project(iMPACT)
最后Create Schematic Symbol,生成逻辑符号图供后序实验使用.

\subsubsection{对原理图进行仿真模拟}
导入仿真激励代码,并且生成波形图检验
\subsubsection*{仿真激励代码}

\begin{lstlisting}[language=C]
   
    #include <stdio.h>

int main()
{
	int i = 0;
	while(1){
	printf("%d",i);
	i++;
	}
	return 0;
}

    
\end{lstlisting}



\subsection{用MyMC14495实现数码管显示}

\section{实验结果与分析}

\section{讨论与心得}

\section{Bonous}

\end{document}