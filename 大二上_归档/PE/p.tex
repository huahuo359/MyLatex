\documentclass{article}
\usepackage[UTF8]{ctex}
\usepackage{pythonhighlight}

% Language setting
% Replace `english' with e.g. `spanish' to change the document language
\usepackage[english]{babel}
\usepackage{float}
% Set page size and margins
% Replace `letterpaper' with `a4paper' for UK/EU standard size
\usepackage[letterpaper,top=2cm,bottom=2cm,left=3cm,right=3cm,marginparwidth=1.75cm]{geometry}

% Useful packages
\usepackage{amsmath}
\usepackage{graphicx}
\usepackage[colorlinks=true, allcolors=blue]{hyperref}

\title{创业启程讲座感想}
\author{雷远航  \  学号:3210105807}

\begin{document}

\maketitle

\begin{abstract}
设计与认知
\end{abstract}

\section*{基于“设计”与“认知”的讲座感想}
在创业启程的几次讲座中,我对应方天教授的“设计驱动式创新”和神策数据创始人、CEO桑文锋的“从码农到独角兽CEO,逆袭背后的认知升级方法论”这两次的讲座有比较深的印象,因此这次的观后感主要围绕着这两次的讲座展开,这两场讲座的关键词分别是”设计”和“认知”,对创业的过程起到了很好的指引作用。

应放天教授的讲座围绕着”设计”的这一主题,设计是创新创业赋能产业的主战场,通过“艺术、技术、商业、用户、文化”这五个方面进行好的整合才能得到好的设计,如果我们在创新创业的过程中能够很好的将这五个方面良好的整合在一起,那么我们就很可能拥有了一次与世界同步的机会,应教授在讲座中向我们提出了现在全球世界创新的4i趋势,他们分别是:Industrial 工业设计、Integration交叉融合、International 国际合作、Innovation 创新精神.这也是对我们进行创新创业活动的一个方向的指引,如果我们能够把握住这样的4i趋势那么也会离成功的创业更进一步,同时在讲座中教授向我们介绍了延展智能的概念,将智能理解为一种基本的分布式现象,每个角色都是延展智能的一部分,将人-机-环境三者相互融合进行自己的创新设计,同时教授以视觉思维和参数化构件的工业设计开发平台为例,介绍了一些这些方面的成功案例,这些成功的企业给了我们很大的启示作用,通过一个优秀的设计来带动产业的发展,要学会打通创意设计和技术之间的隔阂,如果我们能够将自己的创意和先进的技术相互结合,并且可以通过先进的技术来带动好的创意,来实现一个好的创新,真正实现“设计驱动式创新”,我们要认识到设计是技术商品化的重要桥梁,设计是增品种、提品质、创品牌的重要战略,形式创新、功能创新、服务创新是设计的主要特征,如果缺少技术,那么产品就会失去价值,如果缺少设计,产品就会失去生命,在我们要进行未来的创新创业活动时,这样的观点是值得我们去学习和借鉴的,我的主修专业是计算机科学与技术,一些技术类的活动是我们所擅长的,但真正进入创新创业的活动时,如果只是一味的去沉浸于技术,而不进行一些好的设计那么也很难取得成功,技术和设计进行有机的结合才能为成功的创业做好铺垫。设计经济的基础理论框架由思维、创意、经济、价值原理构成,当我们具备了好的设计,做好了前期的铺垫之后便可以真正的走出象牙塔,将自己的想法和思考真正的投入到实践之中,好的设计是我们将想法变成实际的第一步,如果我们能够在设计的过程中有自己的好的想法,这样才可能带来成功的创业,虽然真正的创业的过程中会面临很多的问题和不同的困难,要想取得真正的成功也是需要多种的因素相互结合在一起才是可以的,但应教授的”设计驱动式创新”的观点在我们将要进行创业的启程阶段给了我们一个很好的指引。

在“从码农到独角兽CEO,逆袭背后的认知升级方法论”,这场讲座主要围绕着“认知”展开,正确的认知产生动力,错误的认知产生阻力,对于问题如果能有一个好的、前瞻性的认知,进行前瞻性的规划那么便会为我们未来的创业奠定一个好的基础,一个创业成功的企业、企业家他们往往能具备一个不同于当下时代的认知,不随波逐流,有自己的独到的认知,这种认知也许会在他们当时所处的时代背景下不是一个受到大众认可的认知,当在时间的检验之下可以看到他们的认知的正确性,以互联网企业为例,马云、雷军等人便及时抓住了机遇,凭借着自身对于互联网形式的正确认知最终才有了今天这样的业绩,可见一个好的认知在创新创业的过程中起着一个十分重要的关键性的作用,同时我们的认知要在发展的过程中及时进行调整,我们的认知并不一定能保证是完全正确的,通过一些经历和新的经验,我们的认识可能会有不同,甚至我们曾经的认识可能是错误的,那么我们便要对曾经的认知及时进行调整,才不至于误入歧途,作为神策数据的创始人桑文锋,向我们讲述了他所创办的神策的认知来给我们带来一些启示,通过好的认知才带来了如今他的企业的成功,在一个创新创业的组织当中,如果想要进行认知的升级,一方面要有CEO的认知升级,这种认知的升级从而传递到整个组织当中,通过组织的讨论和决策从而对CEO的决策进行完善和提升,最终会带来整个企业的认知的升级。认知在我们进行创新创业的起到的是一个引领作用,这种引领作用是十分重要的,他为我们的创业的过程中起到了一个方向指引的作用,如果没有一个正确的方向的指引,那么我们的发展必然是漫无目的的,甚至是错误的,那么这样就很难取得成功。

在创业启程的这两场讲座中我有了很大的受益,这两场讲座讲述了在创业的过程的两个重要的因素”设计”和”认知”,同时我也认识到了这两个要素在创业的起步阶段的重要作用,通过一个好的认知我们可以规划出一个正确的方向,并通过这个方向不断进行自己的努力和创造,从而能够在创业的起步阶段规划好自己的方向,另一方面我们可以通过对自己的认知进行好的设计,将自己的想法落实,将想法实际化,带来企业的真正的发展,但是我们也要认识到,创业的过程并非轻而易举的,要有很多的要素相互结合起来,同时也会面临很多的问题和困难,都要及时进行分析和调整,最终我们才会在创业的过程中真正启程。

\end{document}