\documentclass[a4paper, 12pt]{article}
% 可以选择的类型有:article , report , book ...

% 可以设置标题自动编号的等级
\setcounter{secnumdepth}{2}

\setcounter{tocdepth}{2}  % 设置目录层次深入到`\subsection`级

\usepackage{sectsty}
\sectionfont{\centering}  % 可以使标题居中

% 段落进行首行锁进
% \setlength{\parindent}{2em}


%\usepackage[colorlinks=true]{hyperref}  % 为文档中的章节引用自动添加链接

\usepackage[T1]{fontenc}
\usepackage[utf8]{inputenc}
\usepackage{palatino} 

% \usepackage[T1]{fontenc}
% \usepackage[utf8]{inputenc}
\usepackage{palatino} 
% \sectionfont{\fontfamily{phv}\fontseries{b}\fontsize{11pt}{20pt}\selectfont} %一级标题字体格式设置
% \subsectionfont{\fontfamily{phv}\fontseries{b}\fontsize{11pt}{20pt}\selectfont} %二级标题字体格式设置
% \subsubsectionfont{\fontfamily{phv}\fontseries{b}\fontsize{11pt}{20pt}\selectfont} %三级标题字体格式设置

% 使用这个包可以进行中文的插入
\usepackage{xeCJK}

% 进行页眉、页脚的设置
\usepackage{fancyhdr}
\pagestyle{fancy}
\lhead{}
\chead{}
\rhead{\bfseries latex-cookbook} %页眉内容
\lfoot{From: Latex} %页脚内容
\cfoot{To: Hua Huo } %页脚内容
\rfoot{\thepage} %在页脚处给出页码
\renewcommand{\headrulewidth}{0.4pt}
\renewcommand{\footrulewidth}{0.4pt}


% 用于图片插入的包
\usepackage{graphicx}


% 插入代码的一些配置
\usepackage{listings}
\usepackage{color}
\definecolor{codegreen}{rgb}{0,0.6,0}
\definecolor{codegray}{rgb}{0.5,0.5,0.5}
\definecolor{codepurple}{rgb}{0.58,0,0.82}
\definecolor{backcolour}{rgb}{0.95,0.95,0.92}

\lstdefinestyle{mystyle}{
    backgroundcolor=\color{backcolour},   
    commentstyle=\color{codegreen},
    keywordstyle=\color{magenta},
    numberstyle=\tiny\color{codegray},
    stringstyle=\color{codepurple},
    basicstyle=\ttfamily\footnotesize,
    breakatwhitespace=false,         
    breaklines=true,                 
    captionpos=b,                    
    keepspaces=true,                 
    numbers=left,                    
    numbersep=5pt,                  
    showspaces=false,                
    showstringspaces=false,
    showtabs=false,                  
    tabsize=2
}

\lstset{style=mystyle}


% 插入伪代码的配置

\usepackage[linesnumbered, boxed]{algorithm2e}
\usepackage{amsmath, amsfonts}

% 标题、日期、作者信息在前面
\title{Latex Cookbook}
\author{Author}
\date{2023.3.5}


\begin{document}

%\tableofcontents
% 可用于目录的生成


\maketitle


\begin{abstract}
This is a Latex Cookbook
\end{abstract}
\textbf{Keywords: keyword1, keyword2, keyword3}  % 设置关键词



\part{Latex Tutorial}

\section{Document Class}

\subsection{Article}

Hello Latexers! This is our first Latex document

% 在标题之下可以使用段落
\paragraph{PA}
Hello, LaTeXers! This is our first LaTeX document.
\subparagraph{Pa1}
This document is our starting point for learning LaTeX and writing with it. It would not be difficult.

可以写中文


\subsection{Text}


Produce \textit{italicized} text. \\      % 生成斜体字的文本

Produce \textbf{bold face} text. \\       % 生成粗体字的文本

Produce \textsc{small caps} text. \\      % 生成小型大写字母的文本

Produce \texttt{typewriter font} text. \\ % 生成打字机字体的文本



Generate \underline{underlined} text. \\     % 生成带下划线的文本(使用\underline命令)

% Generate \uline{underlined} text. \\         % 生成单下划线的文本(使用\uline命令)

% Generate \uuline{double underlined} text. \\ % 生成单下划线的文本

% Generate \uwave{wavy underlined} text. \\   % 生成波浪线的文本


\subsection{List}

\begin{itemize}

    \item Item 1 % 条目1
    \item Item 2 % 条目2
    
    \end{itemize}
    排序列表的使用方法为:
    \begin{enumerate}
    
    \item Item 1 % 条目1
    \item Item 2 % 条目2
    
    \end{enumerate}
    阐述性列表的使用方法为:
    \begin{description}
    
    \item Item 1 % 条目1
    \item Item 2 % 条目2
    
    \end{description}


\subsection{Section}

% 我们可以插入一个文本框
\fbox{
    \parbox{0.8\linewidth}{
    	In \LaTeX, we can use fbox and parbox to put a box around multiple lines. In this case, we set the linewidth as 0.8.
    }
}




\subsection{Insert Picture}



\begin{figure}[htbp]
%\centering
    \includegraphics[width = 0.5\textwidth]{1.png}
    \caption{A beautiful butterfly.}

\end{figure}



\subsection{Code}

\begin{lstlisting}[language = python]
    import numpy as np
    
    x = np.random.rand(4)
    print(x)
    \end{lstlisting}
   

    This is a simple example:

    \IncMargin{1em}
    \begin{algorithm}
    \SetKwInOut{Input}{Input}
    \SetKwInOut{Output}{Output}
    \caption{Inner product of vectors}
    \Input{$\boldsymbol{x},\boldsymbol{y}\in\mathbb{R}^{n}$}
    \Output{$c$}
    $c=0$\;
    \For{$i=1$ \KwTo $n$}{
    $c=c+x_iy_i$\;
    }
    \end{algorithm}

\end{document}